\section{Pipeline-arkitektur}

Vi startede ud fra vores enkeltcyklusarkitektur fra G1, som vi dog har omstruktureret en del
eftersom vi har fundet smartere måder at lave det på i logisim. Vi tilføjede først de ekstra
instruktioner, og herefter lavede vi de relevante pipeline-registre, en forwarding unit og 
en hazard detection unit som subcircuits.  

\subsection{Instruktioner}
De ekstra instruktioner vi har tilføjet er:
\begin{itemize}
\item R-type: jr, jral
\item J-type: j, jal
\end{itemize}

I alt understøtter vi 2 instruktioner der ikke er en del af opgaven (nor, jral, bne), men som kun kræver ekstra
linjer i en PLA. 

\subsection{Pipeline-registre}

\subsection{Typer af hazards}
I bogen gennemgås tre typer hazard: Data hazards, control hazards og
strukturelle hazards. Som der også gennemgås i bogen er der i mips gjort en del
ud af at forsøge helt at undgå strukturelle hazards, hvorfor disse ikke er
relevante for vores projekt.

\subsection{Data hazards}

\subsection{Control hazards}
Control hazards forekommer ved branches. Da vi ikke på forhånd ved om vi skal hoppe
eller fortsætte ved en branch, risikerer vi at instruktioner der allerede er i vores
pipeline ikke skal udføres. Det går for langsomt at stalle indtil branchen er færdig,
og den optimale løsning, branch prediction, er for avanceret til vores opgave. Derfor
smider vi bare de ugyldige instruktioner væk, hvis det viser sig at et branch bliver taget.
I vores arkitektur forekommer control hazards også ved jumps, da vi ikke holder styr på hvor
der bliver hoppet til.

Pipelinen bliver clearet ved at sætte ClearPipe til 1, som sætter registrene OP i IF/ID, Branching, Memory og Register
ID/EX, samt registrene Branching, Memory og RegWrite i EX/MEM til 0. Derudover sætter den PC til værdien
fra jump- eller branchinstruktionen. 

\subsection{Tests}
Vi har testet på tre måder:
\begin{itemize}
\item Vi har lavet en {\emph masse} uformelle tests mens vi udviklede
diagrammet, hvilket var med til at finde de fleste bugs. Hertil brugte vi
hovedsageligt små testfiler samt vores testkode fra forrige opgave.
\item Da vi var nogenlunde sikre på funktionaliteten kørte vi vores {\tt
primes.asm} hvorved vi nogle få yderligere bugs.
\item Endelig kørte vi den testkode som vi fik af vores instruktor (koden er
vedlagt som et bilag). Da vi kørte denne kode virkede den i første forsøg.
\end{itemize}
