\section{Pipeline-arkitektur}

Vi startede ud fra vores enkeltcyklusarkitektur fra G1, som vi dog har omstruktureret en del
eftersom vi har fundet smartere måder at lave det på i logisim. Vi tilføjede først de ekstra
instruktioner, og herefter lavede vi de relevante pipeline-registre, en forwarding unit og 
en hazard detection unit som subcircuits.  

\subsection{Instruktioner}
De ekstra instruktioner vi har tilføjet er:
\begin{itemize}
\item R-type: jr, jral
\item J-type: j, jal
\begin{itemize}

I alt understøtter vi 2 instruktioner der ikke er en del af opgaven (nor, jral, bne), men som kun kræver ekstra
linjer i en PLA. 

\subsection{Pipeline-registre}

\subsection{Forwarding unit}

\subsection{Hazard detection unit}


\subsection{Tests}
Vi har testet på tre måder:
\begin{itemize}
\item Vi har lavet en {\emph masse} uformelle tests mens vi udviklede
diagrammet, hvilket var med til at finde de fleste bugs. Hertil brugte vi
hovedsageligt små testfiler samt vores testkode fra forrige opgave.
\item Da vi var nogenlunde sikre på funktionaliteten kørte vi vores {\tt
primes.asm} hvorved vi nogle få yderligere bugs.
\item Endelig kørte vi den testkode som vi fik af vores instruktor (koden er
vedlagt som et bilag). Da vi kørte denne kode virkede den i første forsøg.
\end{itemize}
