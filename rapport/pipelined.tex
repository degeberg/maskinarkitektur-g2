\section{Pipeline-arkitektur}

Vi startede ud fra vores enkeltcyklusarkitektur fra G1, som vi dog har omstruktureret en del
eftersom vi har fundet smartere måder at lave det på i logisim. Vi tilføjede først de ekstra
instruktioner, og herefter lavede vi de relevante pipeline-registre, en forwarding unit og 
en hazard detection unit som subcircuits.  

\subsection{Instruktioner}
De ekstra instruktioner vi har tilføjet er:
\begin{itemize}
\item R-type: jr, jral
\item J-type: j, jal
\end{itemize}

I alt understøtter vi 2 instruktioner der ikke er en del af opgaven (nor, jral, bne), men som kun kræver ekstra
linjer i PLAerne i Control og ALU Control. 

\subsection{Pipeline-registre}

Vi har implementeret de pipeline-registre der er omtalt i bogen ved hjælp af
subcircuits i Logisim: IF/ID, ID/EX, EX/MEM, MEM/WB. Hvert register modtager
de informationer der skal bruges i senere pipeline-stadier.

IF/ID, ID/EX og EX/MEM modtager en ClearPipe-bit der indikerer om visse
registre i pipeline-registerbanken skal nulstilles. IF/ID og ID/EX modtager
endvidere en StallPipe-bit, der i IF/ID forhindrer skrivning og i ID/EX sætter
alle control-bits til 0.

\subsection{Typer af hazards}
I bogen gennemgås tre typer hazard: Data hazards, control hazards og
strukturelle hazards. Som der også gennemgås i bogen er der i mips gjort en del
ud af at forsøge helt at undgå strukturelle hazards, hvorfor disse ikke er
relevante for vores projekt.

\subsection{Data hazards}
Der er som udgangspunkt to typer data hazards som er relevante for dette projekt:

Den ene hazard er når der skal bruges et resultat fra ALU'en, før at dette resultat endnu er skrevet
til registerfilen. Dette løses ved hjælp af forwarding.

Den anden hazard er når der skal bruges et resultat fra hukommelsen i en af de
to efterfølgende instruktioner. Hvis det forekommer i instruktionen lige efter
kan dette ikke løses ved hjælp af forwarding, hvorfor at pipelinen stalles en
enkelt cykel, men hvis resultatet først skal bruges to instruktioner senere kan
det løses udelukkende ved hjælp af forwarding.

Forwarding er konkret implementeret præcis som beskrevet i bogen: Der tjekkes
for om der bruges de samme registre, og om disse registre ikke er register {\tt
\$0} - hvis ikke de er, så bruges det overførte resultat i stedet for det hentet
fra registerfilen. Der kan dog bådes overføres resultat fra både {\tt 
MEM}-stadiet og {\tt WB}-stadiet.

Den anden hazard kan som beskrevet kun løses ved at stalle pipelinen i en enkelt
cykel. Dette er konkret implementeret ved at der tjekkes for om instruktionen 
som er i {\tt EX}-stadiet læser fra hukommelsen og skriver til et register som 
skal bruges af instruktionen i {\tt ID}-stadiet. Hvis det er tilfældet, så 
forhindres at værdierne i {\tt PC} og {\tt IF/ID} bliver ændret (medmindre der 
også er en control hazard, hvorved de i stedet bliver clearet) og
control-bitsene i {\tt ID/EX} bliver clearet.

Det skal desuden nævnes, at vi har løst at registerfilen ikke kan se ændringer fra
samme clockcyklus, ved at sætte den til at opdatere på falling edge af clocken.
Dermed behøver vi ikke at forwarde over registerfilen.

\subsection{Control hazards}
Control hazards forekommer ved branches. Da vi ikke på forhånd ved om vi skal hoppe
eller fortsætte ved en branch, risikerer vi at instruktioner der allerede er i vores
pipeline ikke skal udføres. Det går for langsomt at stalle indtil branchen er færdig,
og den optimale løsning, branch prediction, er for avanceret til vores opgave. Derfor
smider vi bare de ugyldige instruktioner væk, hvis det viser sig at et branch bliver taget.
I vores arkitektur forekommer control hazards også ved jumps, da vi ikke holder styr på hvor
der bliver hoppet til.

Pipelinen bliver clearet ved at sætte ClearPipe til 1, som sætter registrene OP i IF/ID, Branching, Memory og Register
ID/EX, samt registrene Branching, Memory og RegWrite i EX/MEM til 0. Derudover sætter den PC til værdien
fra jump- eller branchinstruktionen. 

\subsection{Tests}
Vi har testet på tre måder:
\begin{itemize}
\item Vi har lavet en {\emph masse} uformelle tests mens vi udviklede
diagrammet, hvilket var med til at finde de fleste bugs. Hertil brugte vi
hovedsageligt små testfiler samt vores testkode fra forrige opgave.
\item Da vi var nogenlunde sikre på funktionaliteten kørte vi vores {\tt
primes.asm} hvorved vi nogle få yderligere bugs.
\item Endelig kørte vi den testkode som vi fik af vores instruktor (koden er
vedlagt som et bilag). Da vi kørte denne kode virkede den i første forsøg.
\end{itemize}
