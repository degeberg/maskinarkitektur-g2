\section{Pipeline-arkitektur}

Vi tog vores enkeltcyklusarkitektur fra G1 og indførte de relevante
pipelineregistre som subcircuits i Logisim. Herefter indførte vi en
forwarding unit og en hazard detection unit.

Jumps er delvist blevet implementeret, men fungerer ikke korrekt. Vi er ikke
helt sikre på om stalling virker korrekt. Forwarding burde dog virke.

Den måde vi har testet arkitekturen er ved at skrive de enkelte instruktioner
vi ønskede at teste for derefter at tjekke de enkelte wires og registre og se
om de havde de rigtige værdier.

%Bits fra Control:
%
%[SignedIMM]:
%    Bit 0:Er den eventuelle immediate-værdi signed eller unsigned (1=signed)
%[ALU]:
%    Bit 0: ALUSrc (kommer argument 2 til ALU fra Immediate?)
%    Bit 1-2: 00 = lw/sw, vil gerne have add operation
%             01 = bne/beq, vil gerne have sammenligning
%             10 = Regn det da selv ud ved at bruge funct!
%             11 = Immediate operation specificeret af IMM-OP fra instruktionen
%[Branching]:
%    Bit 0: Kan det forekomme, at der skal branches?
%    Bit 1-2: 00 = j/jal
%             01 = jr/jral
%             10-11 = beq/bne
%    Bit 3: Skal der linkes?
%[Memory]:
%    Bit 0: MemRead, skal der læses fra hukommelsen?
%    Bit 1: MemWrite, skal der skrives til hukommelsen?
%    Bit 2: MemToReg, skal der sendes noget fra memory til register?
%[RegisterWriting]:
%    Bit 0: RegDst, er der specificeret dst register, eller er der kun to
%           registre?
%    Bit 1: Skal der skrives til et register?


%\subsection{Tests}
%Vi har hovedsageligt testet manuelt: Vi har kigget på en bestemt instruktion og
%manuelt set den køre hele vejen igennem pipelinen og verificeret at alle
%relevante signaler er sendt som forventet.
%
%Skriv noget om hvordan vi har testet data hazards. Control hazards var trivielle
%at teste: Lad en køre igennem, og se manuelt at den clearer alle de relevante registre.
