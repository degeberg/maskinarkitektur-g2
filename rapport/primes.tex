\section{Primtalsgenerator i MIPS assemblerkode}
\subsection{Implementation}
Vi har i implementationen af {\tt largest\_prime} valgt at gemme {\tt n}, {\tt
i} og {\tt p} i henholdsvis {\tt \$s0}, {\tt \$s1} og {\tt \$s2}, idet vi kom ud
for at alle disse variable skulle gemmes på tværs af et funktionskald (linje 24
i c-kildekoden). Ved at kigge på {\tt mul} kunne vi dog have valgt at bruge {\tt
\$t[0-2]} i stedet, idet {\tt mul} ikke overskriver disse registre, og på denne
måde have undgået at skulle gemme værdierne på forhånd. Dette ville dog kunne
have givet problemer, hvis implementationen af {\tt mul} senere ændrede sig.

Flere steder i løbet af koden har vi forsøgt at optimere på den, for at få den
til at køre på færre instruktioner eller undgå spildcykler. Dette udvises eksempelvis
ved at vi de fleste steder har optimeret {\tt j} instruktionerne væk fra vores
loops.

\subsection{Test}
Vi testede først vores kode i MARS og fik på den måde
fjernet hvad vi formoder er alle bugs i koden. Dette blev gjort ved at køre
koden igennem instruktion for instruktion og verificerede at alle registre
indeholdt hvad de burde.
